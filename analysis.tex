\refsec{Analysis}{analysis}
\PARstart{N}{ow} that the foundation for the analysis has been laid down in terms of all components used in the proposed design, it is possible to determine the complexity and performance of the complete design. First the complexity and performance for each cycle will be determined for the neuron, activation function and bias neuron implementations. Since this analysis is dependent on some design choices not yet made, a number of variables are introduced to express these choices. These variables are as follows:
\begin{itemize}
  \bfit{$n$} The number of neurons per layer in a single chip, including the bias neuron.
  \bfit{$m$} The number of layers in a single chip.
  \bfit{$w_{addr} = \log_{2}(n)$} The width in bits of the constant part of the neuron addresses.
  \bfit{$w_{cid}$} The width in bits of the \signal{CID} signal, maximally $2^{w_{cid}}$ chips can be combined.
  \bfit{$w_{a} = w_{cid} + w_{addr}$} The width in bits of the complete neuron address.
  \bfit{$w_{d}$} The width in bits of the data paths between neurons.
  \bfit{$w_{da} \leq 2w_{d} + \lceil\log_{2}(n)\rceil$} The number of bits used as input to the activation function.
\end{itemize}

\refsubsec{Neuron analysis}{aneuron}
\PARstart{T}{he} complexity of a single neuron can be established simply by counting the number of occurrences of each component and adding their respective transistor counts. Since a multiplication of two $w_{d}$-bit numbers yields a $2w_{d}$-bit result, the data path from the multiplier to the accumulator is $2w_{d}$ bits wide. The whole accumulator and inhibition data path is $2w_{d} + \lceil\log_{2}(n)\rceil$ bits wide, since this is the number of bits necessary to fully capture an $n$ times accumulation of $2w_{d}$-bit numbers. The highest order $\lceil\log_{2}(n)\rceil$ bits are kept at zero on the \signal{B} input of \texttt{alu\_add}. Implementing the neuron in this way is a design decision made to enable most flexibility and precision. Other options would include making sure that \texttt{alu\_add} clips the output to a certain width and does not overflow the number of bits available. The \signal{d\_o} signal is also $2w_{d} + \lceil\log_{2}(n)\rceil$ bits wide, but the full precision is only used during the inhibition cycle. Only the highest order $w_{da}$ bits are used by the activation function, where most often $w_{da} = w_{d}$ will be chosen, since the precision in the activation function input and output are equal in that case. This leads to a total transistor count for a single neuron as shown in \Cref{eqn:neuron}.

\begin{custeqn}{*}{neuron}
  4 \cdot &\f{T}{g}{bufn}(w_{a}) +                            &\text{(Address \texttt{bufn})}\\
  2 \cdot &\f{T}{g}{bufn}(w_{d}) +                            &\text{(Forward \& backward input \texttt{bufn})}\\
  2 \cdot &\f{T}{t}{regcf}(2^{w_{a}}, w_{d}) +                &\text{(Weight \texttt{regfc})}\\
  2 \cdot &\f{T}{g}{mux}(w_{d}) +                             &\text{(Forward \& backward data \texttt{mux})}\\
          &\f{T}{g}{mul}(w_{d}) +                             &\text{(\texttt{alu\_mul})}\\
          &\f{T}{g}{add}(2w_{d} + \lceil\log_{2}(n)\rceil) +  &\text{(\texttt{alu\_add})}\\
          &\f{T}{g}{regc}(2w_{d} + \lceil\log_{2}(n)\rceil) + &\text{(\texttt{regc\_ihib\_rst})}\\
  2 \cdot &\f{T}{g}{mux}(2w_{d} + \lceil\log_{2}(n)\rceil) +  &\text{(\texttt{mux} before \texttt{reg\_val} \& \texttt{reg\_prev})}\\
  2 \cdot &\f{T}{g}{reg}(2w_{d} + \lceil\log_{2}(n)\rceil) +  &\text{(\texttt{reg\_val} \& \texttt{reg\_prev})}\\
  2 \cdot &\f{T}{g}{bufn}(2w_{d} + \lceil\log_{2}(n)\rceil) + &\text{(Output \& inhibition input \texttt{bufn})}\\
          &\f{T}{g}{mag}(2w_{d} + \lceil\log_{2}(n)\rceil) +  &\text{(\texttt{alu\_gt})}\\
          &\f{T}{g}{reg}(\lceil\log_{2}(n)\rceil) +           &\text{(\texttt{cnt\_ihib})}\\
          &\f{T}{g}{regc}(\lceil\log_{2}(n)\rceil) +          &\text{(\texttt{regc\_ihib\_k})}\\
          &\f{T}{g}{mag}(\lceil\log_{2}(n)\rceil) +           &\text{(\texttt{alu\_lt})}\\
          &\f{T}{g}{ff} +                                     &\text{(\texttt{ff\_tok})}\\
  6 \cdot &\f{T}{g}{bufp} + \rlap{$
  4 \cdot  \f{T}{g}{NOT} +
  6 \cdot  \f{T}{t}{AND} +
  3 \cdot  \f{T}{t}{OR} = $}&&\\
  \rlap{$95 + (539 + 25w_{d}) \cdot w_{d} + 24w_{a} + (207 + 4w_{a} + \lceil\log_{2}(n)\rceil) \cdot \lceil\log_{2}(n)\rceil +$}&&\\
  \rlap{$(20 + 92w_{d}) \cdot 2^{w_{a}} + (16 + 8\sqrt{2^{w_{a}}}) \cdot \log_{2}(\sqrt{2^{w_{a}}})$}&&\\
\end{custeqn}

The performance of a neuron will be expressed in the gate delay of the critical path during each of the three cycles just as was done with each individual component, since the largest gate delay during any of the cycles will determine the maximum frequency of the final chip. During the forward cycle the input critical path goes through the input buffer, which is enabled by the forward signal, \texttt{regcf\_fwd} followed by \texttt{alu\_mul} and \texttt{alu\_add}, finally arriving at \texttt{reg\_val} and \texttt{reg\_prev}. The total input gate delay will therefore become $\f{D}{d}{bufp} + \f{D}{c}{bufn} + \f{D}{d}{regcf} + \f{D}{d}{mux} + \f{D}{d}{mul}(w_{d}) + \f{D}{d}{add}(w_{d}) + \f{D}{d}{mux} + \f{D}{d}{reg} = 23 + 4w_{d}$. During the forward cycle the output critical path goes from \texttt{reg\_prev} through an output buffer enabled by the output of \texttt{ff\_tok}, leading to an output gate delay of $\f{D}{d}{bufp} + \f{D}{d}{ff} + \f{D}{c}{bufn} = 7$.

During the backward cycle the input critical path follows a similar path as during the forward cycle, except it passes through \texttt{regcf\_bwd} instead of \texttt{regcf\_fwd}, this does not have any influence on the gate delay however. Since the output critical path is exactly the same as during the forward cycle, also the output gate delay is the same.

During the inhibition cycle the critical path goes


