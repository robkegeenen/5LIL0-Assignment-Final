\documentclass[a4paper, 9pt, conference, onecolumn]{ieeeconf}

\overrideIEEEmargins
\makeatletter
\let\NAT@parse\undefined
\makeatother
\let\labelindent\relax

\usepackage[utf8]{inputenc}
\usepackage{graphicx}
\usepackage{listings}
\usepackage{float}
\usepackage{amsmath}
\usepackage{amssymb}
\usepackage{todonotes}
\usepackage[printonlyused]{acronym}
\usepackage{xparse}
\usepackage{hyperref}
\usepackage{cleveref}
\usepackage{enumitem}

\title{\bf
  \LARGE Design and achievability study of a dedicated neural network substrate based on industry standard CMOS technology\\
  \vspace{1ex}
  \large [5LIL0] Final assignment
}
\author{
  Geenen, R. (0843558) \\
  \today
}

\setlist{parsep = 0pt, listparindent = \parindent}
\newcommand{\bfit}[1]{\item{\bf #1:}}
\newcommand{\ifnn}[1]{\if\relax\detokenize{#1}\relax\else}
\newcommand{\f}[3]{#1_{\text{\ifnn{#2}#2,\fi#3}}}
%%begin novalidate
\NewDocumentEnvironment{custtab}{mmm}{
  \begin{table}
    \begin{center}
      \begin{tabular}{#1}
}{
      \end{tabular}
      \caption{#3}
      \label{tab:#2}
    \end{center}
  \end{table}
}
%%end novalidate
\newcommand{\signal}[1]{\texttt{$\text{#1}$}}
\newcommand{\signali}[1]{\texttt{$\overline{\text{#1}}$}}

\acrodef{CMOS}{Complementary Metal Oxide Semiconductor}
\acrodef{VLSI}{Very Large Scale Integration}
\acrodef{OE}{Output Enable}

\begin{document}
\maketitle

%\begin{abstract}
%    This is the abstract
%\end{abstract}
%\acresetall

\section{Analysis foundation}
In order to successfully asses the complexity and performance characteristics of the proposed implementation, the complexity and performance characteristics of the underlying components need to be established. Complexity will be measured in terms of transistor count, since that is the industry leading way to indicate complexity of digital \ac{VLSI} circuits. Using the transistor count to measure circuit complexity therefore allows easy comparisons between existing designs and the proposed design.

Furthermore an indicator of the circuit performance is needed. The critical path of a component or circuit indicates the highest number of transistor gates a signal has to pass through before reaching a stabilizing feed-back circuit such as a latch or register. The longest critical path in a circuit combined with the node size\todo[inline]{reference mfr node size} of the manufacturing process determines the maximum clock frequency at which the circuit will be able to operate. The total gate delay of the longest critical path is therefore a useful circuit performance indicator that is agnostic to the manufacturing node used. For this reason the gate delay has been chosen as the metric to be used for assessing circuit performance.

Both the performance and complexity metrics of all blocks used in the proposed design are listed in \Cref{tab:component_char}. Next to the trivial digital gates, a large number of more complex blocks are employed in the proposed design. The reasoning behind the listed metrics will be outlined below.

\begin{itemize}
  \bfit{bufp}
    The permanent buffer is used to buffer signals that need to be distributed to a large number of other components so that the total load does not exceed the input drive strength. The permanent buffer does not need an \ac{OE} input because it is always active, therefore it can simply be composed of two inverters in series.
  \bfit{bufn}
    An $n$-bits tri-state buffer with \ac{OE} input can be built by duplicating the single tri-state buffer circuit\cite[Figure 4]{asakura2003tri} $n$ times and sharing the \signal{OE} and \signali{OE} lines between all bits. Each bit will require an inverter before the input, since the tri-state buffer is itself inverting the signal. An additional inverter should be added to reduce the load on the incoming \signal{OE} line and to make it an active-high signal. This means that $\f{T}{g}{NOT} + 4 = 6$ transistors are needed per bit together with an additional two inverters, leading to a total of $\f{T}{g}{bufn}(n) = 4 + 6n$ transistors. The gate delay will be $\f{D}{d}{bufn} = \f{D}{d}{NOT} + 1 = 2$ gates for the data and $\f{D}{c}{bufn} = 2 \cdot \f{D}{g}{NOT} + 1 = 3$ gates for the control signal.
  \bfit{mux}
    An $n$-bits 2-to-1 multiplexer can be constructed from by duplicating the single 2-to-1 multiplexer circuit\cite[Figure 4]{7375632} $n$ times and sharing the \signal{s} and \signali{s} lines between all bits. The multiplexer will therefore require three NAND gates per bit and an additional inverter, leading to a total transistor count of $\f{T}{g}{mux}(n) = \f{T}{g}{NOT} + 3n \cdot \f{T}{t}{NAND} = 12n + 2$. The data signal only has to pass through two NAND gates, therefore the data gate delay will be $\f{D}{d}{mux} = 2 \cdot \f{D}{d}{NAND} = 2$, while the control signal will need to also pass through an inverter first, leading to a gate delay of $\f{D}{c}{mux} = 2 \cdot \f{D}{d}{NAND} + \f{D}{d}{NOT} = 3$.
  \bfit{dec}
    An $n$-to-1 decoder takes an integer number $x$ of up $n$ bits in size and outputs high on its $x$\textsuperscript{th} output, all other outputs remain low. For arbitrary $n$ it can be built by extending the scheme of an 8-to-1 decoder\cite{74x138} to $n$ inputs. Since no enable inputs are necessary it can be deduced that the $n$-to-1 decoder will require two inverters per input and $2^{n}$ $n$-input NAND gates. This leads to a total transistor count of $\f{T}{g}{dec}(n) = 2n \cdot \f{T}{g}{NOT} + 2^{n} \cdot \f{T}{g}{NAND}(n) = (4 + 2^{n+1}) \cdot n$. Each signal has to pass through two inverters and one NAND gate, causing the total gate delay to be $\f{D}{d}{dec} = 2 \cdot \f{D}{d}{NOT} + \f{D}{d}{NAND} = 3$.
  \bfit{ff}
    An edge-triggered D-type flip-flop with preset and clear can be built by combining $6$ $3$-input NAND gates\cite{74x74}, two additional inverters will be included to transform the preset and clear inputs into active-high inputs. This brings the total transistor count to $\f{T}{g}{ff} = 6 \cdot \f{T}{g}{NAND}(3) + 2 \cdot \f{T}{g}{NOT} = 40$. The flip-flop is the basic signal stabilizing block, meaning its output will remain stable from the start of a clock cycle until the end of a clock cycle, and the output value will only change on a clock edge. However the gate delay for a flip-flop is still important because the signal will need to traverse to the output of the flip-flop before it is kept stable by the internal feedback loop. Considering this the gate delay for the data and clock inputs is $\f{D}{d}{ff} = 2 \cdot \f{D}{d}{NAND} = 2$ since these signals only have to propagate through two NAND gates before reaching the output. The preset and clear signal will additionally have to traverse an inverter, so their gate delay will become $\f{D}{c}{ff} = \f{D}{d}{NOT} + 2 \cdot \f{D}{d}{NAND} = 3$.
  \bfit{reg/regc}
    An $n$-bit register (reg) can be made by configuring $n$ D-type flip-flops in parallel. For $1 \leq i \leq n$ the \signal{D\textsubscript{i}} signals are the $n$-bit parallel input to the register while the \signal{Q\textsubscript{i}} signals are the $n$-bit parallel output from the register. The \signali{Q\textsubscript{i}} signals will be left unconnected, while the preset, clear and clock signals will be shared among the $n$ flip-flops. Meanwhile an $n$-bit configuration register (regc) can be made in a similar way. However the \signal{D\textsubscript{i}} signals will not be the parallel input, instead they will be connected to the \signal{Q\textsubscript{i}} signal of the previous flip-flop, thereby creating a serial in parallel out shift register\cite{74x165}. For both types of register the total transistor count will be $\f{T}{g}{reg}(n) = \f{T}{g}{regc}(n) = n \cdot \f{T}{g}{ff} = 40n$. The gate delay for both configurations is the gate delay of a single flip-flop, being $\f{D}{d}{reg} = \f{D}{d}{regc} = \f{D}{d}{ff} = 2$ and $\f{D}{c}{reg} = \f{D}{c}{regc} = \f{D}{c}{ff} = 3$.
  \bfit{regcf}
    A naive approach to creating an $n$-bits wide, $r$-register long configuration register file would be to assemble $r$ $n$-bits configuration registers (regc) in parallel and add a $\log_{2}(r)$-to-1 decoder to select a register. Depending on the size of $r$ this will become an extremely large contraption quite quickly, since a large decoder is needed. For larger $r$ it is better to place the registers in a square matrix with sizes of $s = 2^{\lceil\log_{2}(\sqrt{r})\rceil}$. 
    
    So for example to reach $r = 2^{16} = 65536$ registers, as will be required for the activation layers, a square matrix of $2^{\lceil\log_{2}(\sqrt{2^{16}})\rceil} = 2^{\lceil\log_{2}(256)\rceil} = 2^{\lceil8\rceil} = 2^{8} = 256$ by $256$ registers will be constructed, leading to a total of $256 \cdot 256 = 65536$ registers. Clearly the space efficiency of this matrix is maximized when $\log_{2}(\sqrt{r})$ yields an integer number, since then no unused registers are generated in the matrix. In order to select a register, two $\log_{2}(s)$-to-1 decoders are now necessary. Even though now two decoders are required instead of one, this still yields a significant space reduction, since $s$ is significantly lower than $r$, easily compensating for requiring twice as many decoders. In order to actually select a register, each register now needs an associated AND gate to detect if both its row and its column are selected. In that case the associated buffer (bufn) of the register will be enabled, allowing the value of the register to propagate to the output of the register file. The input address bits can simply be split over the two decoders, letting the column decoder handle the lower order bits, while the row decoder handles the higher order bits.
    
    In order to configure the register file, the configuration inputs of the registers will be connected to the configuration outputs of the previous registers, while the clock signals will be shared among all registers, thereby creating one enormous shift register. This can be done in a way matching how the address bits are mapped to the decoders. In the suggested case that will be first connecting all registers on a row and then chaining together the rows. This will ensure that the $i$\textsuperscript{th} value shifted in ends up either in the $i$\textsuperscript{th} or in the $(r - i)$\textsuperscript{th} register, depending on which direction the registers are chained in. This will make the configuration easier in the end.
    
    For creating an $n$-bits wide, $r$-register long configuration register file, $r$ $n$-bits configuration registers, two $\log_{2}(s)$-to-1 decoders, $s \cdot s$ AND gates and $s \cdot s$ $n$-bits buffers are necessary. This leads to a total transistor count as shown in \Cref{eqn:regcf_tcount_gen}.
    \begin{equation}
      \label{eqn:regcf_tcount_gen}
      \begin{aligned}
        \f{T}{g}{regcf}(r, n) &= r \cdot \f{T}{g}{regc}(n) + 2 \cdot \f{T}{g}{dec}(\log_{2}(s)) + s^{2} \cdot (\f{T}{t}{AND} + \f{T}{g}{bufn}(n)) \\
                              &= 40rn + 2 \cdot (4 + 2^{\log_{2}(s) + 1}) \cdot \log_{2}(s) + s^{2} \cdot (10 + 6n) \\
                              &= 40rn + 4 \cdot \log_{2}(s) \cdot (2 + s) + s^{2} \cdot (10 + 6n) \\
                              &= 40rn + 4 \cdot \lceil\log_{2}(\sqrt{r})\rceil \cdot (2 + 2^{\lceil\log_{2}(\sqrt{r})\rceil}) + 2^{2 \cdot \lceil\log_{2}(\sqrt{r})\rceil} \cdot (10 + 6n) \\
      \end{aligned}
    \end{equation}
    %
    Assuming that $\lceil\log_{2}(\sqrt{r})\rceil = \log_{2}(\sqrt{r})$, which is a good idea in order to eliminate space wastage, this can be further reduced to the trivial case as shown in \Cref{eqn:regcf_tcount_triv}. It basically only means actually using all registers that will be generated.
    \begin{equation}
      \label{eqn:regcf_tcount_triv}
      \begin{aligned}
        \f{T}{t}{regcf}(r, n) &= 40rn + 4 \cdot \log_{2}(\sqrt{r}) \cdot (2 + 2^{\log_{2}(\sqrt{r})}) + 2^{2 \cdot \log_{2}(\sqrt{r})} \cdot (10 + 6n) \\
                              &= 40rn + 4 \cdot \log_{2}(\sqrt{r}) \cdot (2 + \sqrt{r}) + r \cdot (10 + 6n) \\
                              &= 46rn + 10r + 4 \cdot (2 + \sqrt{r}) \cdot \log_{2}(\sqrt{r}) \\
      \end{aligned}
    \end{equation}
    %
    The address inputs of the register file will all have to pass trough either one of the two decoders, one AND gate and a buffer before the register value is available at the output, this leads to a data gate delay of $\f{D}{d}{regcf} = \f{D}{d}{dec} + \f{D}{d}{AND} + \f{D}{c}{bufn} = 8$. Since the data is loaded into the register file in the same way as it happens for the configuration register, the control gate delay will be equal, therefore $\f{D}{c}{regcf} = \f{D}{c}{regc} = 3$.
  \bfit{fa}
    A single bit full adder can be designed using a constant $28$ transistors\cite[Figure 11.4(b)]{Weste:2010:CVD:1841628}, therefore $\f{T}{g}{fa} = 28$. In this case the gate delay from all inputs to the sum output is $\f{D}{d}{fa} = 3$, while for the carry output it is $\f{D}{c}{fa} = 2$.
  \bfit{add}
    An $n$-bit carry ripple adder can be designed by cascading $n$ full adders\cite[Section 11.2.2.1]{Weste:2010:CVD:1841628}, which will lead to a total transistor count of $\f{T}{g}{add}(n) = n \cdot \f{T}{g}{fa} = 28n$. The sum output of the highest order bit and the carry output is only present once the carry has propagated through all full adders, which means that only then the output will be steady. This means that the gate delay of the $n$-bit carry ripple adder will be $\f{D}{d}{add}(n) = (n - 1) \cdot \f{D}{c}{fa} + \f{D}{d}{fa} = 2n + 1$. This holds since the highest order bit will suffer from a larger delay in the last full adder than the carry output.
  \bfit{mul}
    A fast implementation of an $n$-bit multiplier has been proposed by K.Z. Pekmestzi\cite{743408}. The proposed non-simplified multiplier is preferred, since it has a more regular structure than the simplified version. Later optimization could switch this to the simplified version of the multiplier, but this has no negative consequences, neither for the transistor count, nor for the gate delay. It is shown that the proposed multiplier has a total transistor count of $\f{T}{g}{mul}(n) = 21n^{2} + 89n - 73$ and a gate delay of $\f{D}{d}{mul}(n) = n \cdot \f{D}{c}{fa} + \f{D}{d}{fa} = 2n + 3$. This holds since K.Z. Pekmestzi summarizes the delay of a full adder in a single figure, while only the last full adder in the chain implies the longer delay of computing the sum, all other full adders only have the carry chain in the critical path, similar to the case of the $n$-bit adder.
  \bfit{lt/gt/mag}
    Less than and greater than, or so called magnitude, comparison can be implemented for $n$-bits inputs by extending the scheme as implemented in the \texttt{74x68x} series of devices\cite[p. 4]{74x682}. Since no equality indication is necessary, the $n$-input NAND gate computing the \signali{P=Q} signal can be ignored. This leaves one inverter per input, one XNOR gate and an inverter per pair of inputs, a series of $n$ AND gates with an increasing number of inputs ranging from $2$ to $n + 1$ and an $n$-input NOR gate. In order to be able to compute all possible inequalities two adaptions possibly need to be made. The \signal{P} and \signal{Q} inputs can be swapped and the \signali{P>Q} can be inverted. For this last step a single final inverter might be necessary. \Cref{tab:682_ineqs} lists the possible combinations of these options and what inequality they provide.
    \begin{custtab}{| c | c | c |}{682_ineqs}{Possible options regarding inequality detection.}
      \hline
                               & Swap?      & Invert?    \\ \hline
      $\text{P} >    \text{Q}$ &            & \checkmark \\ \hline
      $\text{P} \geq \text{Q}$ & \checkmark &            \\ \hline
      $\text{P} \leq \text{Q}$ &            &            \\ \hline
      $\text{P} <    \text{Q}$ & \checkmark & \checkmark \\ \hline
    \end{custtab}

    This leads to a total transistor count of $\f{T}{g}{mag}(n) = (3n + 1) \cdot \f{T}{g}{NOT} + n \cdot \f{T}{g}{XNOR} + \sum_{i = 2}^{n + 1}(\f{T}{g}{AND}(i)) + \f{T}{g}{NOR}(n) = 20 \cdot n + 2 + \sum_{i = 2}^{n + 1}(2 + 2i) = n^{2} + 23n + 2$. Since all input bits need to traverse through an inverter, an XNOR gate, an AND gate, an NOR gate and lastly possibly an inverter, the gate delay will be $\f{D}{d}{mag} = 2 \cdot \f{D}{d}{NOT} + \f{D}{d}{XNOR} + \f{D}{d}{AND} + \f{D}{d}{NOR} = 7$.
\end{itemize}

\newcommand{\eqln}[9]{#1
  &          $\begin{aligned}&\f{T}{g}{#1}#2 = #3\end{aligned}$
  &\ifnn{#5} $\begin{aligned}&\f{T}{t}{#1}#4 = #5\end{aligned}$ \fi
  &          $\begin{aligned}&\f{D}{d}{#1}#6 = #7\end{aligned}$
  &\ifnn{#9} $\begin{aligned}&\f{D}{c}{#1}#8 = #9\end{aligned}$ \fi
  \\\hline
}
\begin{custtab}{| l || l | l || l | l |}{component_char}
{The performance and complexity metrics of all blocks used in the proposed design, where $i$ indicates the number of inputs to certain blocks and $n$ indicates the width of an input in bits for certain blocks.}
\hline
Component & \multicolumn{2}{l||}{Transistor count} & \multicolumn{2}{l|}{Gate delay} \\
          & General case & Trivial case            & Data & Control                   \\\hline\hline
\eqln{NOT}   {}       {2}                       {} {}  {}    {1}      {} {}
\eqln{NAND}  {(i)}    {2i}                      {} {4} {}    {1}      {} {}
\eqln{AND}   {(i)}    {2 + 2i}                  {} {6} {}    {2}      {} {}
\eqln{NOR}   {(i)}    {2i}                      {} {4} {}    {1}      {} {}
\eqln{OR}    {(i)}    {2 + 2i}                  {} {6} {}    {2}      {} {}
\eqln{XNOR}  {}       {12}                      {} {}  {}    {2}      {} {}
\eqln{XOR}   {}       {12}                      {} {}  {}    {2}      {} {}
\eqln{bufp}  {}       {4}                       {} {}  {}    {2}      {} {}
\eqln{bufn}  {(n)}    {4 + 6n}                  {} {}  {}    {2}      {} {3}
\eqln{mux}   {(n)}    {2 + 12n}                 {} {}  {}    {2}      {} {3}
\eqln{dec}   {(n)}    {(4 + 2^{n + 1}) \cdot n} {} {}  {}    {3}      {} {}
\eqln{ff}    {}       {40}                      {} {}  {}    {2}      {} {3}
\eqln{reg}   {(n)}    {40n}                     {} {}  {}    {2}      {} {3}
\eqln{regc}  {(n)}    {40n}                     {} {}  {}    {2}      {} {3}
\eqln{regcf} {(r, n)} {    40rn +
                       \\& 4 \cdot \lceil\log_{2}(\sqrt{r})\rceil \cdot (2 + 2^{\lceil\log_{2}(\sqrt{r})\rceil}) +
                       \\& 2^{2 \cdot \lceil\log_{2}(\sqrt{r})\rceil} \cdot (10 + 6n)}
             {(r, n)} {\\& 46rn + 10r +
                       \\& 4 \cdot (2 + \sqrt{r}) \cdot \log_{2}(\sqrt{r})}
             {} {8} {} {3}
\eqln{fa}    {}       {28}                      {} {}  {}    {3}      {} {2}
\eqln{add}   {(n)}    {28n}                     {} {}  {(n)} {2n + 1} {} {}
\eqln{mul}   {(n)}    {21n^{2} + 89n - 73}      {} {}  {(n)} {2n + 3} {} {}
\eqln{mag}   {(n)}    {n^{2} + 23n + 2}         {} {}  {}    {7}      {} {}
\end{custtab}










\bibliographystyle{unsrt}
\bibliography{main}

\end{document}
